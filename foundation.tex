\section{Foundations of risk management}


\subsection{Overview}

\begin{frame}[allowframebreaks]{Overview}
\begin{block}{Concepts}
\begin{itemize}
\item \textbf{Risk}: uncertainty regarding losses/gain.
\item \textbf{Risk management}: Activities aimed to reduce eliminate potential to incur expected loses.
\item \textbf{Risk trading}: taking risk $\to$ generating incremental gains
\end{itemize}
\end{block}

\begin{block}{Risk management process}
\begin{enumerate}
\item Identify 
\item Quantify and estimate
\item Determine collective effects/cost-benefit analysis
\item Develop risk mitigation strategy
\item Assess performance and amend risk mitigation strategy as needed
\end{enumerate}
\end{block}

\begin{block}{Tools and procedures}
\begin{itemize}
\item Quantitative measures: \textit{e.g.} VaR, Expected shortfall, Exposures\footnote{Will re-visit}  
\item Quantitative assessment: scenario analysis, stress testing, 
\item Enterprise risk management (ERM): integrative approach within an entire entity
\end{itemize}
\end{block}

\begin{block}{Compare}
\begin{itemize}
\item Expected \& unexpected loss
\item Risk \& reward
\end{itemize}
\end{block}

\begin{block}{Types of risks}
	\begin{itemize}
		\item Market risk: \textit{e.g.} IR, FX, equity, commodity price, \textit{etc}
		\item Credit risk: \textit{e.g.} default, bankruptcy, downgrade and settlement
		\item Liquidity risk: \textit{e.g.} funding/trading liquidity risk
		\item Operation risk: non-financial problems; challenging to quantify
		\item Legal and regulatory risk
		\item Business risk: uncertainty in business environment (supply/demand, business strategy)
		\item Strategic risk: associated with large new business investments
		\item Reputation risk
	\end{itemize}
\end{block}

\end{frame}

\begin{frame}[allowframebreaks]{Sample question}

\begin{enumerate}
	\item Examining the impact of a dramatic increase in interest rates on the
value of a bond investment portfolio could be performed using which of the
following tools?
	\begin{enumerate}[I]
		\item Stress testing
		\item Enterprise risk management
	\end{enumerate}
    \begin{enumerate}[A]
    	\item I only
    	\item II only
    	\item both I and II
    	\item Neither I nor II
    \end{enumerate}
	\item In considering the major classes of risks, which risk would best describe
an entity with weak internal controls that could easily be circumvented with a
lack of segregation of duties?
	\begin{enumerate}[A]
		\item Business risk
		\item Legal and regulatory risk
		\item Operational risk
		\item Strategic risk
	\end{enumerate}
\end{enumerate}

\begin{enumerate}
	\item Examining the impact of a dramatic increase in interest rates on the
	value of a bond investment portfolio could be performed using which of the
	following tools?
	\begin{enumerate}[I]
		\item Stress testing
		\item Enterprise risk management
	\end{enumerate}
	\begin{enumerate}[A]
		\item I only
		\item II only
		\item {\color{red} Both I and II}
		\item Neither I nor II
	\end{enumerate}
	\item In considering the major classes of risks, which risk would best describe
	an entity with weak internal controls that could easily be circumvented with a
	lack of segregation of duties?
	\begin{enumerate}[A]
		\item Business risk
		\item Legal and regulatory risk
		\item {\color{red} Operational risk}
		\item Strategic risk
	\end{enumerate}
\end{enumerate}

\end{frame}

\subsection{Corporate risk management}

\begin{frame}[allowframebreaks]{Corporate risk management}{Hedging}
	%hedging
	\begin{block}{Hedging risk exposures -- advantages and disadvantages}
		\begin{itemize}
			\item Theoretical considerations
			\begin{itemize}
				\item Assumptions of hedging: perfect competitive capital markets;
			\end{itemize}
			\item Disadvantages:
			\begin{itemize}
				\item Cost management's focus on core business
				\item Compliance costs
				\item May increase the variability of the firm's earnings
			\end{itemize}
			\item Advantages:
			\begin{itemize}
				\item Lowering cost of capital
				\item Increase shareholders confidence (stable earnings/cash flows)
				\item Controlled financial performance
				\item Operational improvements within a firm
				\item Taxation
			\end{itemize}
		\end{itemize}
	\end{block}

	\begin{block}{Role of board of directors}
		\begin{itemize}
			\item Set the firm's risk appetite
			\item Tools of risk management: e.g. risk tolerance, VaR, stress testing
			\item Risk and return goals
		\end{itemize}
	\end{block}

	\begin{block}{Hedging practice}
		\begin{itemize}
			\item Clarification as to which risks are insurable, hedgeable, noninsurable,
			or nonhedgeable
			\item Can be performed for various risks factors
			\begin{itemize}
				\item Pricing risk (e.g. production companies)
				\item Foreign exchange
				\item Interest rate
			\end{itemize}
		    \item Static vs dynamic hedging strategies 
		    \item Risk management instruments: Exchange traded, OTC
		\end{itemize}
	\end{block}
\end{frame}


\begin{frame}{Sample question}

\begin{enumerate}
	\item Which of the following statements regarding exchange-traded and over-the-counter
	(OTC) financial instruments is correct?
	\begin{enumerate}[A]
		\item {\color{red} There is greater liquidity with exchange-traded financial instruments.}
		\item There is greater customization with exchange-traded financial instruments.
		\item There is greater price transparency with OTC financial instruments.
		\item There is credit risk by either of the counterparties inherent in exchange-traded instruments.
	\end{enumerate}
\end{enumerate}

\end{frame}

\subsection{Corporate governance and risk management}

\begin{frame}[allowframebreaks]{Corporate governance and risk management}
\begin{block}{Best practice in corporate governance}
	\begin{itemize}
		\item Members with knowledge of the
		firm's business and industry
		\item Board watches out for the interests of all stakeholders
		\item Aware of any agency risks and takes steps to reduce them 
		\item Board maintains its independence from management 
		\item Board should consider the introduction of a chief risk officer
	\end{itemize}
\end{block}

\begin{block}{Best practice in risk management}
	\begin{itemize}
		\item Focus on the economic performance over accounting performance
		\item Robust risk management process within the firm
		\item Ethics committee; Risk-adjusted performance
		\item Board should have a risk committee in place
	\end{itemize}
\end{block}

\begin{block}{Role of the board of directors in governance}
	\begin{itemize}
		\item The firm's risk management policies
		\item The firm's periodic risk management reports
		\item The firm's appetite and its impact on business strategy
		\item The firm's internal controls
		\item The firm's financial statements and disclosures
		\item The firm's related parties and related party transactions
		\item Any audit reports from internal or external audits
		\item Corporate governance best practices for the industry
		\item Risk management practices of competitors and the industry
	\end{itemize}
\end{block}

\begin{block}{Concepts}
	\begin{itemize}
		\item Risk appetite: tolerance (especially willingness) to accept risk
		\item Role of Audit committee, risk advisory director, risk management committee and compensation committee
		\item Functional units within a firm 
		\begin{enumerate}
			\item senior management
			\item risk management
			\item trading room management
			\item operations
			\item finance
		\end{enumerate}
	\end{itemize}
\end{block}

\end{frame}

\begin{frame}{Sample question I}

\begin{enumerate}
	\item Which of the following statements regarding the firm's risk appetite and/or its
	business strategy is most accurate?
	\begin{enumerate}[A]
		\item The firm's risk appetite does not consider its willingness to accept risk.
		\item {\color{red} The board needs to work with management to develop the firm's overall strategic
		plan.}
		\item Management will set the firm's risk appetite and the board will provide its
		approval of the strategic plan.
		\item Management should obtain the risk management team's approval once the
		business planning process is finalized.
	\end{enumerate}
\end{enumerate}

\end{frame}

\begin{frame}{Sample question II}

\begin{enumerate}
	\item The various responsibilities surrounding the profit and loss (P\&L) statement
	illustrate the importance of understanding the interdependence of managing risk
	within a firm. Within an investment bank, which functional unit is most likely to
	provide final approval of the P\&L?
	\begin{enumerate}[A]
		\item Finance
		\item Operations
		\item Senior management
		\item {\color{red} Trading room management}
	\end{enumerate}
\end{enumerate}

\end{frame}

\subsection{Enterprise risk management}

\begin{frame}[allowframebreaks]{Enterprise risk management}

\begin{block}{ERM}
	\begin{itemize}
		\item Short comings of tradition risk management
		\item ERM: integrated and centralized risk management framework
	\end{itemize}
\end{block}

\begin{block}{ERM definitions}
	\begin{itemize}
		\item Often defined as a process or activity to manage risks
		\item Committee of Sponsoring Organizations of the Treadway Commission (COSO)
		\item International Organization of Standardization (ISO 3000)
		\item \textit{Risk is a variable that can cause deviation from an expected outcome. ERM is a
		comprehensive and integrated framework for managing key risks in order to achieve business
		objecrives, minimize unexpected earnings volatility, and maximize firm value}
	\end{itemize}
\end{block}

\begin{block}{ERM benefits and costs}
	\begin{itemize}
		\item Increased organizational effectiveness
		\item Better risk reporting
		\item Improved business performance
		\item Can be time-consuming to implement and requires ongoing senior management support
	\end{itemize}
\end{block}

\begin{block}{The role of CRO}
	\begin{itemize}
		\item Responsible for all risks facing a company; overall leadership for ERM
		\item Critical skills: (1) leadership, (2) power of persuasion, (3) ability
		to protect the firm's assets, (4) technical skills to understand all risks, and (5) consulting
		skills to educate the board and business functions on risk management.
	\end{itemize}
\end{block}

\begin{block}{Main components of a strong ERM framework}
	\begin{itemize}
		\item Corporate governance
		\item Line management
		\item Portfolio management
		\item Risk transfer
		\item Risk analytics
		\item Data and technology resources
		\item Stakeholder management
	\end{itemize}
\end{block}

\end{frame}

\begin{frame}{Sample question}

\begin{enumerate}
	\item The basis of enterprise risk management (ERM) is that:
	\begin{enumerate}[A]
		\item risks are managed within each risk unit but centralized at the senior
		management level.
		\item the silo approach to risk management is the optimal risk management strategy.
		\item risks should be managed and centralized within each risk unit.
		\item it is necessary to appoint a chief risk officer to oversee most risks.
	\end{enumerate}
\end{enumerate}

\end{frame}

\begin{frame}{Sample question}

\begin{enumerate}
	\item The basis of enterprise risk management (ERM) is that:
	\begin{enumerate}[A]
		\item {\color{red} risks are managed within each risk unit but centralized at the senior
		management level.}
		\item the silo approach to risk management is the optimal risk management strategy.
		\item risks should be managed and centralized within each risk unit.
		\item it is necessary to appoint a chief risk officer to oversee most risks.
	\end{enumerate}
\end{enumerate}

\end{frame}


\begin{frame}[allowframebreaks]{Risk trading and risk management by banks}

\begin{block}{Optimal level of risk exposure}
	\begin{itemize}
		\item methods:
		\begin{itemize}
			\item Targeting a certain credit rating
			\item Sensitivity or scenario analysis
			\item Analysis of the adverse impacts on the value of
			a bank due to changes in interest rates, foreign exchange rates, inflation, \textit{etc}
		\end{itemize}
		\item goals:
		\begin{itemize}
			\item Maximize shareholder value; while
			\item Satisfying the constraints by regulators
		\end{itemize}
	\end{itemize}
\end{block}

\begin{block}{Adding value by risk management}
	\begin{itemize}
		\item Prevent taking excessive risk
		\item Flexible risk management system may allow the bank to take on profitable risks
		\item Requiring business units to take
		the perspective of the entire bank when making decisions regarding risks
	\end{itemize}
\end{block}

\begin{block}{Limitations of hedging}
	\begin{itemize}
		\item Risk measurement technology limitations
		\item Hedging limitations
		\item Risk taker incentive limitations
	\end{itemize}
\end{block}

\begin{block}{Challenges in risk profile and performance}
	\begin{itemize}
		\item Limited data exists on how the risk function operates in banks
		\item Risk function characteristics are affected by the
		bank's risk appetite
		\item It is possible that poor performance will occur even in the presence of strong governance
	\end{itemize}
\end{block}

\end{frame}

\subsection{Financial disasters}

\begin{frame}{Financial disasters}

\begin{table}[htbp]
	\centering
	%\caption{Financial disasters}
	{\small \begin{tabular}{|l|l|l|}
		\hline
		\textbf{Firms} & \textbf{Year} & \textbf{Key Reason} \bigstrut\\
		\hline
		Chase Manhattan & 1976  & \multirow{5}[10]{*}{Misleading reporting} \bigstrut\\
		\cline{1-2}    Kidder Peabody & 1992  &  \bigstrut\\
		\cline{1-2}    Barings & 1994  &  \bigstrut\\
		\cline{1-2}    Allied Irish Bank & 1997-2002 &  \bigstrut\\
		\cline{1-2}    UBS   & 1997  &  \bigstrut\\
		\hline
		Scoiete Generale & 2008  & \multirow{3}[6]{*}{Large market movement} \bigstrut\\
		\cline{1-2}    LTCM  & 1994  &  \bigstrut\\
		\cline{1-2}    Metallgesellschaft & 1991  &  \bigstrut\\
		\hline
		Bankers Trust & 1994  & \multirow{2}[4]{*}{Customer conduct} \bigstrut\\
		\cline{1-2}    JP Morgan, Citigroup and Enron & 2001  &  \bigstrut\\
		\hline
	\end{tabular}}
	\label{tab:disasters}
\end{table}

\end{frame}

\begin{frame}{Sample question I}

\begin{enumerate}
	\item Hedging models at Long-Term Capital Management accounted for the:
	\begin{enumerate}[I]
		\item British law tax changes and large Japanese bank warrants.
		\item Incorrect valuation of long-dated options on equity baskets and inappropriate
		modeling of other long-dated options.
	\end{enumerate}
	\begin{enumerate}[A]
		\item I only
		\item II only
		\item Both I and II
		\item Neither I nor II
	\end{enumerate}
\end{enumerate}

\end{frame}

\begin{frame}{Sample question II}

\begin{enumerate}
	\item Hedging models at Long-Term Capital Management accounted for the:
	\begin{enumerate}[I]
		\item British law tax changes and large Japanese bank warrants.
		\item Incorrect valuation of long-dated options on equity baskets and inappropriate
		modeling of other long-dated options.
	\end{enumerate}
	\begin{enumerate}[A]
		\item I only
		\item II only
		\item Both I and II
		\item {\color{red}Neither I nor II}
	\end{enumerate}
\end{enumerate}

\end{frame}