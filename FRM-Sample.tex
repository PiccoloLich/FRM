\documentclass[10pt]{beamer}



\usetheme[
%%% option passed to the outer theme
%    progressstyle=fixedCircCnt,   % fixedCircCnt, movingCircCnt (moving is deault)
]{Feather}

% If you want to change the colors of the various elements in the theme, edit and uncomment the following lines

% Change the bar colors:
%\setbeamercolor{Feather}{fg=red!20,bg=red}

% Change the color of the structural elements:
%\setbeamercolor{structure}{fg=red}

% Change the frame title text color:
%\setbeamercolor{frametitle}{fg=blue}

% Change the normal text color background:
%\setbeamercolor{normal text}{fg=black,bg=gray!10}

%-------------------------------------------------------
% INCLUDE PACKAGES
%-------------------------------------------------------

\usepackage[utf8]{inputenc}
\usepackage[english]{babel}
\usepackage[T1]{fontenc}
\usepackage{helvet}
\usepackage{bigstrut}
\usepackage{threeparttable}

%-------------------------------------------------------
% DEFFINING AND REDEFINING COMMANDS
%-------------------------------------------------------

% colored hyperlinks
\newcommand{\chref}[2]{
	\href{#1}{{\usebeamercolor[bg]{Feather}#2}}
}

\AtBeginSection[]
{
	\begin{frame}<beamer>
	\frametitle{Outline}
	\tableofcontents[sections = \thesection]
	\end{frame}
}

%-------------------------------------------------------
% INFORMATION IN THE TITLE PAGE
%-------------------------------------------------------

\title[] % [] is optional - is placed on the bottom of the sidebar on every slide
{ % is placed on the title page
	\textbf{FRM Level 1 Lecture Notes}
}

\subtitle[FRM Level 1 Notes]
{
	\textbf{v. 1.0.0}
}

\author[]
{      Bin Guo \\
	%{\ttfamily lilqna.v@gmail.com}
}

\institute[]
{
	%Tip Top\\
	
	%there must be an empty line above this line - otherwise some unwanted space is added between the university and the country (I do not know why;( )
}

\date{\today}

%-------------------------------------------------------
% THE BODY OF THE PRESENTATION
%-------------------------------------------------------

\begin{document}
	
	%-------------------------------------------------------
	% THE TITLEPAGE
	%-------------------------------------------------------
	
	{\1% % this is the name of the PDF file for the background
		\begin{frame}[plain,noframenumbering] % the plain option removes the header from the title page, noframenumbering removes the numbering of this frame only
		\titlepage % call the title page information from above
\end{frame}}


\begin{frame}{Content}{}
\tableofcontents
\end{frame}


%-------------------------------------------------------
\section{Introduction}
%-------------------------------------------------------
\subsection{The FRM exam}
\begin{frame}{Introduction}{The FRM exam}
%-------------------------------------------------------

\begin{itemize}
	\item International professional certification offered by \textbf{GARP} (The Global Association of Risk Professionals). 
	\item A certificate focus on risk management, 2 levels:
	\begin{itemize}
		\item \textbf{Part 1}: tools used to assess financial risk : Foundations of Risk Management, Quantitative Analysis, Financial Markets and Products, Valuation and Risk Models
		\item \textbf{Part 2}: Measurement and Management of: Market Risk, Credit Risk, Operational and Integrated Risk; Current Issues in Financial Markets
	\end{itemize} 
	\item Scoring and results:
	\begin{itemize}
		\item All multiple choice questions
		\item No penalties for wrong answers
		\item Passing scores determined by FRM committee ($\sim 50\%$)
		\item Exam results emailed six weeks after the exam, quartile results
	\end{itemize}   
\end{itemize}
\end{frame}

\begin{frame}{Introduction}{Part 1 exam}
\begin{table}
\centering
\caption{Part 1 exam contents and weights}
\begin{threeparttable}[htbp]
	\begin{tabular}{||r|l|r|r||}
		\hline
		\hline
		\multicolumn{1}{||l|}{Book} & Knowledge Domains & \multicolumn{1}{l|}{Weight} & \multicolumn{1}{l||}{\# Questions} \bigstrut\\
		\hline
		1     & Foundations of Risk Management & 20\%  & 20 \bigstrut\\
		\hline
		2     & Quantitative Analysis & 20\%  & 20 \bigstrut\\
		\hline
		3     & Financial Markets and Products & 30\%  & 30 \bigstrut\\
		\hline
		4     & Valuation and Risk Models & 30\%  & 30 \bigstrut\\
		\hline
		\hline
	\end{tabular}
	\begin{tablenotes}
		\item [a] 4 hours exam time 
	\end{tablenotes}
\end{threeparttable}
\label{tab:weights}
\end{table}
\end{frame}

\subsection{Key to success}
\begin{frame}{Introduction}{Key to success}
\begin{block}{Plan and practice}
	\begin{itemize}
		\item Begin studying early
		\item Study plan and stick to the plan
		\item \textbf{Practice exams}
	\end{itemize}
\end{block}

\begin{block}{Recourses}
	\begin{itemize}
		\item Official books
		\item Schweser 
		\item bbs.pinggu.org
	\end{itemize}
\end{block}
\end{frame}

\subsection{Study plan}
\begin{frame}{Introduction}{Study plan}
\begin{itemize}
	\item 2017 Level 1 exam: Sat. May 20 (4 weeks from now)
	\item Take your own background into consideration when planning
	\item Foundations of risk management: 
	\begin{itemize}
		\item 14 Key knowledge points; 20 questions; Mainly concepts/memorizations
	\end{itemize} 
	\item Quantitative analysis: 
	\begin{itemize}
		\item 15 key knowledge points; 20 questions; Mainly technical questions
		\item Can be straightforward if you are from a technical background
	\end{itemize}
	\item Financial markets and products: 
	\begin{itemize}
		\item 30 key points; 30 questions; concepts and technical;
		\item Familiar for students from finance background
	\end{itemize}	
	\item Valuation and risk models:
	\begin{itemize}
		\item 17 key points; 30 questions; concepts and technical;
	\end{itemize}
	\item Practice
	\begin{itemize}
		\item High importance! Should allocate at least 1.5-2 weeks
		\item Past exam questions; GARP mock exam; Concept checker
		\item {\color{red} This class will focus on summarizing knowledge points and solving exam questions}
	\end{itemize}
	
	
\end{itemize}
\end{frame}

   %introduction section
\section{Foundations of risk management}


\subsection{Overview}

\begin{frame}[allowframebreaks]{Overview}
\begin{block}{Concepts}
\begin{itemize}
\item \textbf{Risk}: uncertainty regarding losses/gain.
\item \textbf{Risk management}: Activities aimed to reduce eliminate potential to incur expected loses.
\item \textbf{Risk trading}: taking risk $\to$ generating incremental gains
\end{itemize}
\end{block}

\begin{block}{Risk management process}
\begin{enumerate}
\item Identify 
\item Quantify and estimate
\item Determine collective effects/cost-benefit analysis
\item Develop risk mitigation strategy
\item Assess performance and amend risk mitigation strategy as needed
\end{enumerate}
\end{block}

\begin{block}{Tools and procedures}
\begin{itemize}
\item Quantitative measures: \textit{e.g.} VaR, Expected shortfall, Exposures\footnote{Will re-visit}  
\item Quantitative assessment: scenario analysis, stress testing, 
\item Enterprise risk management (ERM): integrative approach within an entire entity
\end{itemize}
\end{block}

\begin{block}{Compare}
\begin{itemize}
\item Expected \& unexpected loss
\item Risk \& reward
\end{itemize}
\end{block}

\begin{block}{Types of risks}
	\begin{itemize}
		\item Market risk: \textit{e.g.} IR, FX, equity, commodity price, \textit{etc}
		\item Credit risk: \textit{e.g.} default, bankruptcy, downgrade and settlement
		\item Liquidity risk: \textit{e.g.} funding/trading liquidity risk
		\item Operation risk: non-financial problems; challenging to quantify
		\item Legal and regulatory risk
		\item Business risk: uncertainty in business environment (supply/demand, business strategy)
		\item Strategic risk: associated with large new business investments
		\item Reputation risk
	\end{itemize}
\end{block}

\end{frame}

\begin{frame}[allowframebreaks]{Sample question}

\begin{enumerate}
	\item Examining the impact of a dramatic increase in interest rates on the
value of a bond investment portfolio could be performed using which of the
following tools?
	\begin{enumerate}[I]
		\item Stress testing
		\item Enterprise risk management
	\end{enumerate}
    \begin{enumerate}[A]
    	\item I only
    	\item II only
    	\item both I and II
    	\item Neither I nor II
    \end{enumerate}
	\item In considering the major classes of risks, which risk would best describe
an entity with weak internal controls that could easily be circumvented with a
lack of segregation of duties?
	\begin{enumerate}[A]
		\item Business risk
		\item Legal and regulatory risk
		\item Operational risk
		\item Strategic risk
	\end{enumerate}
\end{enumerate}

\begin{enumerate}
	\item Examining the impact of a dramatic increase in interest rates on the
	value of a bond investment portfolio could be performed using which of the
	following tools?
	\begin{enumerate}[I]
		\item Stress testing
		\item Enterprise risk management
	\end{enumerate}
	\begin{enumerate}[A]
		\item I only
		\item II only
		\item {\color{red} Both I and II}
		\item Neither I nor II
	\end{enumerate}
	\item In considering the major classes of risks, which risk would best describe
	an entity with weak internal controls that could easily be circumvented with a
	lack of segregation of duties?
	\begin{enumerate}[A]
		\item Business risk
		\item Legal and regulatory risk
		\item {\color{red} Operational risk}
		\item Strategic risk
	\end{enumerate}
\end{enumerate}

\end{frame}

\subsection{Corporate risk management}

\begin{frame}[allowframebreaks]{Corporate risk management}{Hedging}
	%hedging
	\begin{block}{Hedging risk exposures -- advantages and disadvantages}
		\begin{itemize}
			\item Theoretical considerations
			\begin{itemize}
				\item Assumptions of hedging: perfect competitive capital markets;
			\end{itemize}
			\item Disadvantages:
			\begin{itemize}
				\item Cost management's focus on core business
				\item Compliance costs
				\item May increase the variability of the firm's earnings
			\end{itemize}
			\item Advantages:
			\begin{itemize}
				\item Lowering cost of capital
				\item Increase shareholders confidence (stable earnings/cash flows)
				\item Controlled financial performance
				\item Operational improvements within a firm
				\item Taxation
			\end{itemize}
		\end{itemize}
	\end{block}

	\begin{block}{Role of board of directors}
		\begin{itemize}
			\item Set the firm's risk appetite
			\item Tools of risk management: e.g. risk tolerance, VaR, stress testing
			\item Risk and return goals
		\end{itemize}
	\end{block}

	\begin{block}{Hedging practice}
		\begin{itemize}
			\item Clarification as to which risks are insurable, hedgeable, noninsurable,
			or nonhedgeable
			\item Can be performed for various risks factors
			\begin{itemize}
				\item Pricing risk (e.g. production companies)
				\item Foreign exchange
				\item Interest rate
			\end{itemize}
		    \item Static vs dynamic hedging strategies 
		    \item Risk management instruments: Exchange traded, OTC
		\end{itemize}
	\end{block}
\end{frame}


\begin{frame}{Sample question}

\begin{enumerate}
	\item Which of the following statements regarding exchange-traded and over-the-counter
	(OTC) financial instruments is correct?
	\begin{enumerate}[A]
		\item {\color{red} There is greater liquidity with exchange-traded financial instruments.}
		\item There is greater customization with exchange-traded financial instruments.
		\item There is greater price transparency with OTC financial instruments.
		\item There is credit risk by either of the counterparties inherent in exchange-traded instruments.
	\end{enumerate}
\end{enumerate}

\end{frame}

\subsection{Corporate governance and risk management}

\begin{frame}[allowframebreaks]{Corporate governance and risk management}
\begin{block}{Best practice in corporate governance}
	\begin{itemize}
		\item Members with knowledge of the
		firm's business and industry
		\item Board watches out for the interests of all stakeholders
		\item Aware of any agency risks and takes steps to reduce them 
		\item Board maintains its independence from management 
		\item Board should consider the introduction of a chief risk officer
	\end{itemize}
\end{block}

\begin{block}{Best practice in risk management}
	\begin{itemize}
		\item Focus on the economic performance over accounting performance
		\item Robust risk management process within the firm
		\item Ethics committee; Risk-adjusted performance
		\item Board should have a risk committee in place
	\end{itemize}
\end{block}

\begin{block}{Role of the board of directors in governance}
	\begin{itemize}
		\item The firm's risk management policies
		\item The firm's periodic risk management reports
		\item The firm's appetite and its impact on business strategy
		\item The firm's internal controls
		\item The firm's financial statements and disclosures
		\item The firm's related parties and related party transactions
		\item Any audit reports from internal or external audits
		\item Corporate governance best practices for the industry
		\item Risk management practices of competitors and the industry
	\end{itemize}
\end{block}

\begin{block}{Concepts}
	\begin{itemize}
		\item Risk appetite: tolerance (especially willingness) to accept risk
		\item Role of Audit committee, risk advisory director, risk management committee and compensation committee
		\item Functional units within a firm 
		\begin{enumerate}
			\item senior management
			\item risk management
			\item trading room management
			\item operations
			\item finance
		\end{enumerate}
	\end{itemize}
\end{block}

\end{frame}



     %fundation of risk management



%\begin{frame}{Installation}{Source files}
%%-------------------------------------------------------
%
%\begin{block}{}
%The theme contains 4 source files:
%\begin{itemize}
%\item {\tt beamercolorthemeFeather.sty}
%\item {\tt beamerouterthemeFeather.sty}
%\item {\tt beamerinnerthemeFeather.sty}
%\item {\tt beamerthemeFeather.sty}
%\end{itemize}
%\end{block}
%\end{frame}
%
%%-------------------------------------------------------
%\subsection{Local and Global installation}
%\begin{frame}{Installation}{Local and Global installation}
%%-------------------------------------------------------
%The theme can be installed for \textbf{local} or \textbf{global} use.
%\pause
%\begin{block}{Local Installation}
%\begin{itemize}    
%\item Local installation is the simplest way of installing the theme. 
%\item You need to placing the 4 source files in the same folder as your presentation. When you download the theme, the 4 theme files are located in the {\tt local} folder.
%\end{itemize}
%\end{block}
%
%\begin{block}{Global Installation}
%\begin{itemize}
%\item If you wish to make the theme globally available, you must put the files in your local latex directory tree. The location of the root of the local directory tree depends on your operating system and the latex distribution. 
%\item Detailed steps on how to proceed installation under various operating systems can be found at Beamer documentation.
%\end{itemize}
%\end{block}
%\end{frame}
%
%
%%-------------------------------------------------------
%\subsection{Required Packages}
%\begin{frame}{Installation}{Required Packages}
%%-------------------------------------------------------
%
%For using the Feather Theme you will need the Bemaer class installed and the following 2 packages
%\begin{itemize}
%\item TikZ\footnote{TikZ is a package for creating beautiful graphics. Have a look at these \chref{http://www.texample.net/tikz/examples/}{online examples} or the \chref{http://tug.ctan.org/tex-archive/graphics/pgf/base/doc/generic/pgf/pgfmanual.pdf}{pgf user manual}.}
%\item calc
%\end{itemize}
%Due to the fact that the packages are very common they should be included in your latex distribution in the first place.
%\end{frame}
%
%%-------------------------------------------------------
%\section{User Interface}
%\subsection{Loading the Theme and Theme Options}
%\begin{frame}{User Interface}{Loading the Theme and Theme Options}
%%-------------------------------------------------------
%
%\begin{block}{The Presentation Theme}
%The Feather Theme can be loaded in a familiar way. In the reamble of your {\tt tex} file you must type\\ \vspace{5pt} 
%{\tt \textbackslash usetheme[<options>]\{Feather\}}\\ \vspace{5pt} 
%The presentation theme loads the inner, outer and color Feather theme files and passes the {\tt <options>} on to these files.
%\end{block}
%\begin{block}{The Inner and Outher Themes}
%If you wish you can load only the inner, or the outher theme directly by\\ \vspace{5pt} 
%{\tt \textbackslash useinnertheme\{Feather\}} (and it has no options)\\ \vspace{5pt} 
%{\tt \textbackslash useoutertheme[<options>]\{Feather\}} (it has one option)\\
%\hspace{20pt}{\tt progressstyle=\{fixedCircCnt or movingCircCnt\}} \\
%\begin{itemize}
%\item which set how the progress is illustrated;
%\item the value {\tt movingCircCnt} is the default.
%\end{itemize}
%\end{block}
%\end{frame}
%
%\begin{frame}{User Interface}{Loading the Theme and Theme Options}
%
%\begin{block}{The Color Theme}
%Also you can load only the color theme by writing in the preamble of the {\tt tex} file 
%
%\vspace{5pt} 
%
%\begin{itemize}
%\item {\tt \textbackslash usecolortheme\{Feather\}}
%\end{itemize}
%
%\vspace{5pt}
%
%...or to change the colors of the various elements in the theme
%
%\vspace{5pt} 
%\begin{itemize}
%\item Change the bar colors: \\    
%{\tt \textbackslash setbeamercolor \{Feather\}\{fg=<color>, bg=<color>\}}
%
%\vspace{2pt} 
%
%\item Change the color of the structural elements: \\    
%{\tt \textbackslash setbeamercolor\{structure\}\{fg=<color>\}}
%
%\vspace{2pt} 
%
%\item Change the frame title text color:\\
%{\tt \textbackslash setbeamercolor\{frametitle\}\{fg=<color>\}}
%
%\vspace{2pt} 
%
%\item Change the normal text color background:    
%{\tt \textbackslash setbeamercolor\{normal text\}\{fg=<color>, bg=<color>\}}
%\end{itemize}
%\end{block}
%\end{frame}
%
%
%%-------------------------------------------------------
%\subsection{Feather image}
%\begin{frame}{User Interface}{The Feather Background Image}
%%-------------------------------------------------------
%
%\begin{block}{The Feather Background Image}
%\begin{itemize}
%\item In Feather theme, the title page frame and the last frame have the Feather image as the background image. 
%\item The Feather background image can be produced to any frame by wrating on the begining at the choosen frame the following
%\end{itemize} 
%
%\vspace{5pt} 
%
%{\tt \{\textbackslash 1bg\\
%\textbackslash begin\{frame\}[<options>]\{Frame Title\}\{Frame Subtitle\}\\
%\ldots\\
%\textbackslash end\{frame\}\}}
%\end{block}
%\end{frame}
%
%

{\1
\begin{frame}[plain,noframenumbering]
\finalpage{Thank you!}
\end{frame}}

\end{document}