
%-------------------------------------------------------
\section{Introduction}
%-------------------------------------------------------
\subsection{The FRM exam}
\begin{frame}{Introduction}{The FRM exam}
%-------------------------------------------------------

\begin{itemize}
	\item International professional certification offered by \textbf{GARP} (The Global Association of Risk Professionals). 
	\item A certificate focus on risk management, 2 levels:
	\begin{itemize}
		\item \textbf{Part 1}: tools used to assess financial risk : Foundations of Risk Management, Quantitative Analysis, Financial Markets and Products, Valuation and Risk Models
		\item \textbf{Part 2}: Measurement and Management of: Market Risk, Credit Risk, Operational and Integrated Risk; Current Issues in Financial Markets
	\end{itemize} 
	\item Scoring and results:
	\begin{itemize}
		\item All multiple choice questions
		\item No penalties for wrong answers
		\item Passing scores determined by FRM committee ($\sim 50\%$)
		\item Exam results emailed six weeks after the exam, quartile results
	\end{itemize}   
\end{itemize}
\end{frame}

\begin{frame}{Introduction}{Part 1 exam}
\begin{table}
\centering
\caption{Part 1 exam contents and weights}
\begin{threeparttable}[htbp]
	\begin{tabular}{||r|l|r|r||}
		\hline
		\hline
		\multicolumn{1}{||l|}{Book} & Knowledge Domains & \multicolumn{1}{l|}{Weight} & \multicolumn{1}{l||}{\# Questions} \bigstrut\\
		\hline
		1     & Foundations of Risk Management & 20\%  & 20 \bigstrut\\
		\hline
		2     & Quantitative Analysis & 20\%  & 20 \bigstrut\\
		\hline
		3     & Financial Markets and Products & 30\%  & 30 \bigstrut\\
		\hline
		4     & Valuation and Risk Models & 30\%  & 30 \bigstrut\\
		\hline
		\hline
	\end{tabular}
	\begin{tablenotes}
		\item [a] 4 hours exam time 
	\end{tablenotes}
\end{threeparttable}
\label{tab:weights}
\end{table}
\end{frame}

\section{Key to success}
\begin{frame}{Introduction}{Key to success}
\begin{block}{Plan and practice}
\begin{itemize}
\item Begin studying early
\item Study plan and stick to the plan
\item \textbf{Practice exams}
\end{itemize}
\end{block}

\begin{block}{Recourses}
\begin{itemize}
\item Official books
\item Schweser 
\item bbs.pinggu.org
\end{itemize}
\end{block}
\end{frame}

\section{Foundations of risk management}
\subsection{Overview}

\begin{frame}[allowframebreaks]{Overview}
\begin{block}{Concepts}
\begin{itemize}
\item \textbf{Risk}: uncertainty regarding losses/gain.
\item \textbf{Risk management}: Activities aimed to reduce eliminate potential to incur expected loses.
\item \textbf{Risk trading}: taking risk $\to$ generating incremental gains
\end{itemize}
\end{block}

\begin{block}{Types of risk}
\begin{itemize}
\item Market risk: \textit{e.g.} IR, FX, equity, commodity price, \textit{etc}
\item Credit risk: \textit{e.g.} default, bankruptcy, downgrade and settlement
\item Liquidity risk: \textit{e.g.} funding/trading liquidity risk
\item Operation risk
\end{itemize}
\end{block}

\begin{block}{Risk management process}
\begin{enumerate}
\item Identify 
\item Quantify and estimate
\item Determine collective effects/cost-benefit analysis
\item Develop risk mitigation strategy
\item Assess performance and amend risk mitigation strategy as needed
\end{enumerate}
\end{block}

\begin{block}{Tools and procedures}
\begin{itemize}
\item Quantitative measures: \textit{e.g.} VaR, Expected shortfall, Exposures\footnote{Will re-visit}  
\item Quantitative assessment: scenario analysis, stress testing, 
\item Enterprise risk management (ERM): integrative approach within an entire entity
\end{itemize}
\end{block}

\begin{block}{Expected \& unexpected loss}
\begin{itemize}
\item Quantitative measures: \textit{e.g.} VaR, Expected shortfall, Exposures\footnote{Will re-visit}  
\item Quantitative assessment: scenario analysis, stress testing, 
\item Enterprise risk management (ERM): integrative approach within an entire entity
\end{itemize}
\end{block}

\end{frame}